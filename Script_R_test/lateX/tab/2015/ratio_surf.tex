%latex.default(ratio, title = str_c(), rowname = "", rowlabel = "",     where = "ht", longtable = FALSE, col.just = strsplit("l l l",         " ")[[1]], rgroup = c("\\textit{prorata} surfaces+productivit�",         "\\textit{prorata} surfaces"), n.rgroup = table(ratio$ratio_type),     label = "ratio", caption = "Ratio � appliquer sur chacun des axes du secteur 1 (Allier en aval de Langeac, Alagnon et Dore) en fonction de deux m�thodologies de calcul diff�rentes. \\textit{prorata} surfaces ne tient compte que de la disponibilit� des habitats productifs \nsur chacun des axes, \\textit{prorata} surfaces+productivit� tient compte � la fois du ratio d'habitats productifs sur chacun des axes mais �galement de la productivit� de ces axes via un coefficient pond�rateur\nissu de l'analyse des p�ches �lectriques sur chacun de ces axes. $r_{river}$\\_75-03 �tant le ratio � appliquer � chaque cours d'eau sur la p�riode 1975-2003 (c'est-�-dire avant l'ouverture de Grand Pont sur l'Alagnon) et $r_{river}$\\_04-14 le ratio � appliquer depuis 2004.",     caption.lot = "Ratio � appliquer sur chacun des axes du secteur 1 pour r�partir les juv�niles produits dans le secteur 1 (Vichy-Langeac + Allier + Dore)",     file = str_c(tabwd, "ratio_surf.tex"))%
\begin{table}[ht]
\caption[Ratio � appliquer sur chacun des axes du secteur 1 pour r�partir les juv�niles produits dans le secteur 1 (Vichy-Langeac + Allier + Dore)]{Ratio � appliquer sur chacun des axes du secteur 1 (Allier en aval de Langeac, Alagnon et Dore) en fonction de deux m�thodologies de calcul diff�rentes. \textit{prorata} surfaces ne tient compte que de la disponibilit� des habitats productifs 
sur chacun des axes, \textit{prorata} surfaces+productivit� tient compte � la fois du ratio d'habitats productifs sur chacun des axes mais �galement de la productivit� de ces axes via un coefficient pond�rateur
issu de l'analyse des p�ches �lectriques sur chacun de ces axes. $r_{river}$\_75-03 �tant le ratio � appliquer � chaque cours d'eau sur la p�riode 1975-2003 (c'est-�-dire avant l'ouverture de Grand Pont sur l'Alagnon) et $r_{river}$\_04-14 le ratio � appliquer depuis 2004.\label{ratio}} 
\begin{center}
\begin{tabular}{llll}
\hline\hline
\multicolumn{1}{l}{}&\multicolumn{1}{c}{Cours d'eau}&\multicolumn{1}{c}{$r_{river}$\_75-03}&\multicolumn{1}{c}{$r_{river}$\_04-14}\tabularnewline
\hline
{\bfseries \textit{prorata} surfaces+productivit�}&&&\tabularnewline
~~&Allier (Vichy-Langeac)&0.66&0.43\tabularnewline
~~&Alagnon&0.14&0.44\tabularnewline
~~&Dore&0.2&0.13\tabularnewline
\hline
{\bfseries \textit{prorata} surfaces}&&&\tabularnewline
~~&Allier (Vichy-Langeac)&0.61&0.46\tabularnewline
~~&Alagnon&0.08&0.3\tabularnewline
~~&Dore&0.31&0.24\tabularnewline
\hline
\end{tabular}\end{center}

\end{table}

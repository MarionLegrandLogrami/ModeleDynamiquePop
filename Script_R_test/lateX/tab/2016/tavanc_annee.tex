%latex.default(tab, title = str_c(), rowname = "", rowlabel = "",     where = "ht", longtable = FALSE, col.just = strsplit("l l l",         " ")[[1]], rgroup = c("2014", "2015", "2016"), n.rgroup = table(tab$annee),     label = "avancement", caption = "", caption.lot = "", file = str_c(tabwd,         "tavanc_annee.tex"))%
\begin{table}[ht]
\caption[]{\label{avancement}} 
\begin{center}
\begin{tabular}{lll}
\hline\hline
\multicolumn{1}{l}{}&\multicolumn{1}{c}{annee}&\multicolumn{1}{c}{avanc}\tabularnewline
\hline
{\bfseries 2014}&&\tabularnewline
~~&2014&Conversion des surfaces productives selon la formule des ERR d�velopp�e sur l'Allier (Minster\&Bomassi, 1999)\tabularnewline
~~&2014&Prise en compte plus fine des surfaces sous influence des d�versements (jusqu'� l'ann�e 2005)\tabularnewline
~~&2014&Mise � jour des donn�es 2012 et 2013\tabularnewline
~~&2014&D�veloppement des projections li�es au r�am�nagement de Pout�s (50\% d'am�lioration et suppression de l'ouvrage)\tabularnewline
\hline
{\bfseries 2015}&&\tabularnewline
~~&2015&Diff�rence de fitnsess entre les juv�niles issus de reproduction naturelle et les juv�niles d�vers�s (bibliographie). Le travail n'a pas conclu sur l'utilit� de modifier les hypoth�ses ant�rieures du mod�le\tabularnewline
~~&2015&Mise � jour des donn�es 2014\tabularnewline
~~&2015&D�veloppement d'un sc�nario de gestion (simulation � 20 ans) concernant la suppression des impacts � la d�valaison dans les ouvrages hydro�lectriques situ�s dans le secteur du mod�le dynamique de population\tabularnewline
\hline
{\bfseries 2016}&&\tabularnewline
~~&2016&Ajout d'une quatri�me zone dans le mod�le en extrayant l'Alagnon du secteur aval du mod�le qui comprenait jusque l� l'Allier en aval de Langeac + la Dore + l'Alagnon\tabularnewline
~~&2016&Mise � jour des donn�es 2015\tabularnewline
\hline
\end{tabular}\end{center}

\end{table}
